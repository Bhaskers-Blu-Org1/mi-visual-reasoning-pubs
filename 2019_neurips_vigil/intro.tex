\section{Introduction}

Fluid Intelligence:

IQ may be viewed as a composite comprising multiple elements: In many theories of intelligence, a distinction is made between fluid and crystallized intelligence (8). Fluid intelligence comprises the set of abilities involved in coping with novel environments and especially in abstract reasoning; crystallized intelligence is the product of the application of these processes. Fluid intelligence is often measured by tests such as figural analogy, classification, and matrix problems, whereas crystallized intelligence is measured by tests of vocabulary and general information (9). 
~\cite{sternberg2008increasing}

Cognitively, gF is thought to be related to metacognition6(knowing about and reflecting upon one’s own ongoing mentalprocesses) and to working memory4,5,7–9(the active maintenanceof domain-specific information plus domain-general attention-al or ‘executive’ control of ongoing processing). One componentof attentional control is the ability to overcome interference thatwould otherwise disrupt performance by compromising taskgoals or information held active in working memory. Individualdifferences in gF are most pronounced in behavioral measureswhen attentional control is required4,8,10. For this reason, gF isthought to be related to attentional control specifically~\cite{gray2003neural}


In recent years there has been substantial progress in sys-tems  that  can  find  factual  answers  in  text,  starting  withIBM’s Watson system~\cite{ferrucci2010building}

, and now with high-performing neural systems that can answer short ques-tions provided they are given a text that contains the answer e.g.~\cite{wang2018glue}


AI  has  achieved  remarkable  mastery  over  games  such  asChess, Go, and Poker, and evenJeopardy!, but the rich variety of standardized exams has remained a landmark chal-lenge.   Even  in  2016,  the  best  AI  system  achieved  merely 59.3\% on an 8th Grade science exam challenge (Schoenicket al., 2016).




Playing Atari Games~\cite{mnih2015human}


Despite several successes across many domains Deep learning~\cite{lecun2015deep} still struggles with

learning algoritms
learning reasoning~\cite{graves2016hybrid}


Wingrad Scheme challenge~\cite{levesque2012winograd}

visual reasoning~\cite{mogadala2019trends} - datasets such as COG~\cite{yang2018dataset} and 
SVQA (Synthetic Video Question Answering)~\cite{song2018explore}


ARISTO project~\cite{clark2019f} - based on RoBERTa~\cite{liu2019roberta} contextual word embeddings

transformer-based solutions~\cite{vaswani2017attention} using self-attention


“The current neural network approaches will find it difficult to determine which combinations of ‘later’, ‘earlier’, ‘more’, and ‘less’ constitute ‘increase’ and which constitute ‘decrease,'” Davis says. “Neural networks have no inherent idea of magnitude or of time.”
~\cite{davis2016write}

bAbI~\cite{weston2015towards}

Visual Dialog~\cite{das2017visual} - the same, they keep the whole history of the dialog in memory

\begin{itemize}
\item bAbI:  MemNets~\cite{weston2014memory} have access to the whole story at once
\item the same goes to SoftPats~\cite{haurilet2019s} - they build graph per frame and then frame number is treated as one dimensions, so at the end the \textit{Traveler} can access all of them at the same time 
\item The paper~\cite{le2019learning} focuses on SVQA and TGIF-QA -  they access all frames at once, i.e. cut the video into clips, process each frame with CNNs and then aggregate feature representations of equal-size clips obtained by a temporal attention mechanism. So in fact the model has access to all frames all the time.
\end{itemize}
so the time aspect is really... not dealt with?

Additionally, in~\cite{song2018explore} the authors introduced a large-scale dataset caled SVQA (Synthetic Video Question Answering) consisting of (Total QA pairs: 83160/11760/23760 and Total Videos 8400/1200/2400).
As "using all frames is time-consuming. Thus we divide each video into clips (segments) of 16 frames, with 80\% overlap between successive clips (segments)" and "We extract feature from each clip and aggregate features of all clips from one video to form a sequential video representation." -- which means that they identified the problem that you "cannot extract features from all frames" at the beginning and pass that to the model. But instead of proposing a solution that will deal with the video on per-frame basis, they "cheated". ;)

IMPORTANT: \textbf{we do not have any explicit assumptions when it comes to number of frames/length of the movie/number of distractionts}, so there is no need for cutting video into cuts etc.

\subsection{Contributions}

\begin{itemize}
\item \textbf{Time aspect}:
\begin{itemize}
\item Learning the temporal association - grounding the time-related words with meaning
\item Learning the concept of time
\item time context being by-product of gates
\end{itemize}

\item Visual Grounding:
\begin{itemize}
\item Learning complex, multi-step reasoning that involves grounding of words and visual representations/objects
\item 
\end{itemize}
\item \textbf{Selective Attention Memory}:
\begin{itemize}
\item updating the memory content only with relevant visual information depending on the temporal context
\item content based and location based addressing for reading and writing
\item new memory interface/gating designed in such a way enabling the model to control the flow of current visual information and content of the memory in a selective way

\end{itemize}
\end{itemize}







