\section{Introduction}

Integration of vision and language in deep neural network models allows the system to learn joint representations of objects, concepts, and relations.  Potentially, this approach can address Harnad's "symbol grounding problem" \cite{harnad2003symbol} and lead to visually grounded language learning.
%In addition to multimodal learning, visual grounding of language is another opportunity that the AI community is excited about, especially in the context of Harnad's "symbol grounding problem" \cite{harnad2003symbol}. 

Starting with the Visual Question Answering (VQA) dataset \cite{}, many tasks that integrate vision and language have emerged in the past several years\cite{mogadala2019trends}.  As a departure from the simpler vision tasks like classification and object detection, visual reasoning has become the core emphasis in these tasks. Visual reasoning mainly deals with the comparison of object attributes, counting and other relational questions.

In contrast to VQA, which comes with static image and question pairs, another emerging direction is Video QA \cite{}, which introduces the aspect of time.  In addition to spatial reasoning, video input provides an opportunity to work on temporal relations and reasoning.  As evident from human cognition, attention and memory are the key competencies required to solve these problems, and unsurprisingly, the AI research is rapidly growing in these areas.

The ability to deal with time can pose a challenge for natural language processing (NLP) as well; for instance, in question answering (QA) and dialog applications.  The current NLP solutions, in certain problem settings, can work around this challenge by processing the entire text input and reason over it multiple times using attention \cite{vaswani2017attention} or other mechanisms. For example, solutions to the bAbI reasoning task (e.g. Memory Networks \cite{weston2015towards}), typically involve processing the supporting facts all at once, which reside in memory and available to provide answers. Similarly, in Visual Dialog~\cite{das2017visual} the system keeps the whole history of the dialog in memory.  In real-time dialog or video QA, there may not be such an opportunity to have the question and entire input all at once in the beginning.  

Video reasoning datasets such as SVQA (Synthetic Video Question Answering)~\cite{song2018explore} and COG \cite{yang2018dataset} have limited number of frames (e.g. 4-16 frames), and therefore manageable by the neural net models to process and represent all of the visual information.  As an example, according to\cite{song2018explore} the authors extracted visual features from each frame and aggregated features of all clips from one video to form a sequential video representation.
These solutions may demonstrate high accuracy, however, generalization to arbitrary length video sequences will be problematic.  Such approaches also differ from the human cognitive strategies.  Humans have the ability to selectively pay attention and store salient items or events in memory based on their goals.  Working memory and episodic memory systems are responsible for this remarkable ability. 

%Fluid Intelligence:
%
%IQ may be viewed as a composite comprising multiple elements: In many theories of intelligence, a distinction is made between fluid and crystallized intelligence (8). Fluid intelligence comprises the set of abilities involved in coping with novel environments and especially in abstract reasoning; crystallized intelligence is the product of the application of these processes. Fluid intelligence is often measured by tests such as figural analogy, classification, and matrix problems, whereas crystallized intelligence is measured by tests of vocabulary and general information (9). 
%~\cite{sternberg2008increasing}
%
%Cognitively, gF is thought to be related to metacognition6(knowing about and reflecting upon one’s own ongoing mentalprocesses) and to working memory4,5,7–9(the active maintenanceof domain-specific information plus domain-general attention-al or ‘executive’ control of ongoing processing). One componentof attentional control is the ability to overcome interference thatwould otherwise disrupt performance by compromising taskgoals or information held active in working memory. Individualdifferences in gF are most pronounced in behavioral measureswhen attentional control is required4,8,10. For this reason, gF isthought to be related to attentional control specifically~\cite{gray2003neural}


%In recent years there has been substantial progress in systems  that  can  find  factual  answers  in  text,  starting  with IBM’s Watson system~\cite{ferrucci2010building}, and now with high-performing neural systems that can answer short questions provided they are given a text that contains the answer e.g.~\cite{wang2018glue}
%
%
%AI  has  achieved  remarkable  mastery  over  games  such  asChess, Go, and Poker, and evenJeopardy!, but the rich variety of standardized exams has remained a landmark chal-lenge.   Even  in  2016,  the  best  AI  system  achieved  merely 59.3\% on an 8th Grade science exam challenge (Schoenicket al., 2016).
%
%Playing Atari Games~\cite{mnih2015human}
%
%
%Despite several successes across many domains Deep learning~\cite{lecun2015deep} still struggles with
%
%learning algoritms
%learning reasoning~\cite{graves2016hybrid}
%
%
%Wingrad Scheme challenge~\cite{levesque2012winograd}

%visual reasoning~\cite{mogadala2019trends} - datasets such as COG~\cite{yang2018dataset} and 
%SVQA (Synthetic Video Question Answering)~\cite{song2018explore}
%
%
%ARISTO project~\cite{clark2019f} - based on RoBERTa~\cite{liu2019roberta} contextual word embeddings
%
%transformer-based solutions~\cite{vaswani2017attention} using self-attention
%
%
%“The current neural network approaches will find it difficult to determine which combinations of ‘later’, ‘earlier’, ‘more’, and ‘less’ constitute ‘increase’ and which constitute ‘decrease,'” Davis says. “Neural networks have no inherent idea of magnitude or of time.”
%~\cite{davis2016write}





%\begin{itemize}
%\item bAbI:  MemNets~\cite{weston2014memory} have access to the whole story at once
%\item the same goes to SoftPats~\cite{haurilet2019s} - they build graph per frame and then frame number is treated as one dimensions, so at the end the \textit{Traveler} can access all of them at the same time 
%\item The paper~\cite{le2019learning} focuses on SVQA and TGIF-QA -  they access all frames at once, i.e. cut the video into clips, process each frame with CNNs and then aggregate feature representations of equal-size clips obtained by a temporal attention mechanism. So in fact the model has access to all frames all the time.
%\end{itemize}
%so the time aspect is really... not dealt with?
%
%Additionally, in~\cite{song2018explore} the authors introduced a large-scale dataset caled SVQA (Synthetic Video Question Answering) consisting of (Total QA pairs: 83160/11760/23760 and Total Videos 8400/1200/2400).
%As "using all frames is time-consuming. Thus we divide each video into clips (segments) of 16 frames, with 80\% overlap between successive clips (segments)" and "We extract feature from each clip and aggregate features of all clips from one video to form a sequential video representation." -- which means that they identified the problem that you "cannot extract features from all frames" at the beginning and pass that to the model. But instead of proposing a solution that will deal with the video on per-frame basis, they "cheated". ;)
%
%IMPORTANT: \textbf{we do not have any explicit assumptions when it comes to number of frames/length of the movie/number of distractionts}, so there is no need for cutting video into cuts etc.

\subsection{Contributions}
This work have attempted to address the issues mentioned above and propose a new model that can dynamically process video input frame-by-frame, reason over images and remember the salient concepts to answer questions.  We have tested this model on the COG dataset \cite{yang2018dataset} and compared to the baseline results.  Our results indicate that the model is capable of:
%\begin{itemize}
%\item \textbf{Time aspect}:
\begin{itemize}
\item Learning the temporal association - grounding the time-related words with meaning
%\item Learning the concept of time
%\item time context being by-product of gates
%\end{itemize}

\item Learning complex, multi-step reasoning that involves grounding of words and visual representations of objects/attributes
%\begin{itemize}
%\item Learning complex, multi-step reasoning that involves grounding of words and visual representations/objects
\item Selectively control the flow of information to and from the memory to answer questions
%\end{itemize}
%\item \textbf{Selective Attention Memory}:
%\begin{itemize}
\item Updating the memory content only with relevant visual information depending on the temporal context
%\item content based and location based addressing for reading and writing
%\item new memory interface/gating designed in such a way enabling the model to control the flow of current visual information and content of the memory in a selective way

\end{itemize}
%\end{itemize}







