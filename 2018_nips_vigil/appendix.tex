\appendix

\section{Description of datasets}

Most of the VQA datasets contain ambiguities and have strong linguistic biases that allow a model to learn answering strategies that exploit those biases, without reasoning about the visual input~\cite{Santoro2017ASN}.
The CLEVR dataset~\cite{johnson2017clevr} was developed to address those issues and come back to the core challenge of visual QA which is reasoning.
CLEVR contains images of 3D-rendered objects, such as spheres and cylinders.
Each image is associated with a number of highly compositional questions that fall into different categories.
Those categories fall into 5 classes of tasks: Exist, Count, Compare Integer, Query Attribute and Compare Attribute. 
For example, query attribute questions may ask “What is the color of the sphere?”, while compare attribute questions may ask “Is the cube the same material as the cylinder? ”.
The CLEVR dataset consists of:
\begin{itemize}
\item 	A training set of 70k images and 700k questions,
\item	A validation set of 15k images and 150k questions,
\item	A test  set of 15k images and 150k questions about objects,
\item	Answers, scene graphs and functional programs for all train and val images and questions.
\end{itemize}
Each object present in the scene, aside of position, is characterized by a set of four attributes:
\begin{itemize}
\item 2 sizes: large, small,
\item 3 shapes: square, cylinder, sphere,
\item 2 material types: rubber, metal,
\item 8 color types: gray, blue, brown, yellow, red, green, purple, cyan,
\end{itemize}
resulting in 96 unique combinations.

Along with CLEVR, the authors~\cite{johnson2017clevr} introduced  CLEVR-CoGenT (Compositional Generalization Test, CoGenT in short), with a goal of testing how well the models can learn and generalizer compositional concepts.
This dataset is synthesized in the same way as CLEVR,, but contains two conditions.
As shown in \tableref{tab:cogent_conditions}, in Condition A all cubes are gray, blue, brown, or yellow, whereas all cylinders are red, green, purple, or cyan; in Condition B cubes and cylinders swap color palettes.
Both conditions contain spheres of all colors.
CoGenT thus indicates how a model answers CLEVR questions: by memorizing combinations of traits or by learning disentangled or general representations.
The CoGenT dataset contains:
\begin{itemize}
\item	A training set of 70,000 images and 699,960 questions in Condition A,
\item	A validation set of 15,000 images and 150,000 questions in Condition A,
\item	A validation set of 15,000 images and 149,991 questions in Condition B,
\item	A test set of 15,000 images and 149,980 questions in Condition B,
\item	A test set of 15,000 images and 149,992 questions in Condition B,
\item	Answers, scene graphs and functional programs for all training and validation images and questions.
\end{itemize}

\begin{table}[h!]
	\centering
	\begin{tabular}{cccc}
		\toprule
		Dataset        & Cubes              & Cylinders &  Spheres         \\
		\midrule
		CLEVR   &  any color &  any color        &    any color    \\
		%\midrule
		CLEVR CoGenT A & gray / blue / brown / yellow  & red / green / purple / cyan       &    any color  \\
		CLEVR CoGenT B  & red / green / purple / cyan &   gray / blue / brown / yellow       &      any color  \\
		\bottomrule
	\end{tabular}
	\caption{Colors/shapes combinations present in CLEVR, CoGenT-A and CoGenT-B datasets}
	\label{tab:cogent_conditions}
\end{table}

 
\section{Full MAC and S-MAC comparison}

In \tableref{tab:results_full} we present the full comparison between MAC and S-MAC models.


\begin{table}
	\caption{CLEVR \& CoGenT accuracies for the MAC \& S-MAC models}
	\centering
	\begin{tabular}{ccccCcCc}
		\toprule
		\multirow{2}{*}{Model} & \multicolumn{3}{c}{Training} &  \multicolumn{2}{c}{Fine-tuning} &  \multicolumn{2}{c}{Test} \\
		\cmidrule{2-4} \cmidrule{5-6} \cmidrule{7-8} 
		& Dataset                & Time [h:m] & Acc [\%]          & Dataset & Acc [\%]  & Dataset & Acc [\%] \\
		\midrule
		\multirow{15}{*}{MAC} & \multirow{10}{*}{CLEVR}  & \multirow{10}{*}{30:52}  & \multirow{10}{*}{96.70} & \multirow{4}{*}{--}   & \multirow{4}{*}{--}  & CLEVR    & 96.17          \\
		\cmidrule{7-8} 
		&                        &  &               &     &                                & CoGenT-A    &  96.22   \\
		\cmidrule{7-8} 
		&                        &   &              &     &                               & CoGenT-B   & 96.27  \\
		
		\cmidrule{5-6} \cmidrule{7-8} 
		&                             &                                         &    &   \multirow{2}{*}{CoGenT-A}         &       \multirow{2}{*}{98.06}          & CoGenT-A &  94.60	         \\
		\cmidrule{7-8} 
		&                             &                                         &       &         &                & CoGenT-B &    93.28       \\
		\cmidrule{5-6} \cmidrule{7-8} 
		&                             &                                         &    &   \multirow{2}{*}{CoGenT-B}         &       \multirow{2}{*}{98.16}          & CoGenT-A &  93.02         \\
		\cmidrule{7-8} 
		&                             &                                         &       &         &                & CoGenT-B &    94.44       \\  
		
		\cmidrule{2-4} \cmidrule{5-6} \cmidrule{7-8} 
		& \multirow{5}{*}{CoGenT-A} & \multirow{5}{*}{30:52}     & \multirow{5}{*}{97.02}   &  \multirow{2}{*}{--}  &  \multirow{2}{*}{--}    & CoGenT-A & 96.88         \\
		\cmidrule{7-8} 
		&                             &                                         &       &         &                & CoGenT-B & 79.54          \\
		\cmidrule{5-6} \cmidrule{7-8} 
		&                             &                                         &    &   \multirow{2}{*}{CoGenT-B}         &       \multirow{2}{*}{97.91}          & CoGenT-A &  92.06         \\
		\cmidrule{7-8} 
		&                             &                                         &       &         &                & CoGenT-B &    95.62       \\
		\midrule
		\multirow{15}{*}{S-MAC} & \multirow{10}{*}{CLEVR}  & \multirow{10}{*}{28:30}  & \multirow{10}{*}{95.82} & \multirow{3}{*}{--}   & \multirow{3}{*}{--}  & CLEVR    & 95.29           \\
		\cmidrule{7-8} 
		&                        &  &               &     &                                & CoGenT-A    &  95.47   \\
		\cmidrule{7-8} 
		&                        &   &              &     &                               & CoGenT-B   &  95.58  \\		
		
		\cmidrule{5-6} \cmidrule{7-8} 
		&                             &                                         &    &   \multirow{2}{*}{CoGenT-A}         &       \multirow{2}{*}{97.48}          & CoGenT-A &  93.44         \\
		\cmidrule{7-8} 
		&                             &                                         &       &         &                & CoGenT-B &    92.31       \\
		\cmidrule{5-6} \cmidrule{7-8} 
		&                             &                                         &    &   \multirow{2}{*}{CoGenT-B}         &       \multirow{2}{*}{97.67}          & CoGenT-A &  92.11         \\
		\cmidrule{7-8} 
		&                             &                                         &       &         &                & CoGenT-B &    92.95       \\  		
		
		\cmidrule{2-4} \cmidrule{5-6} \cmidrule{7-8} 
		& \multirow{5}{*}{CoGenT-A}   & \multirow{5}{*}{28:33}   & \multirow{5}{*}{96.09}  &  \multirow{2}{*}{--}  &  \multirow{2}{*}{--}   & CoGenT-A & 95.91          \\
		\cmidrule{7-8} 
		&                             &                                         &     &          &                & CogenT-B & 78.71          \\
		\cmidrule{5-6} \cmidrule{7-8} 
		&                             &                                         &    &   \multirow{2}{*}{CoGenT-B}         &       \multirow{2}{*}{96.85}          & CoGenT-A &  91.24         \\
		\cmidrule{7-8} 
		&                             &                                         &       &         &                & CoGenT-B &    94.55       \\
		\bottomrule
	\end{tabular}
	\label{tab:results_full}
\end{table}

\section{Comparison of generalization capabilities}

In this section we present comparison of MAC and S-MAC with selected state-of-the-art models as reported in~\cite{mascharka2018transparency} \tk{and? FiLM?}


\begin{table}[!h]
	\caption{Generalization capabilities of selected state-of-the-art models}
	\centering
	\begin{tabular}{cCcCcCc}
		\toprule
		Model & \multicolumn{2}{c}{Training} &    \multicolumn{2}{c}{Fine-tuning} &   \multicolumn{2}{c}{Test} \\		
		\cmidrule{2-3} \cmidrule{4-5}\cmidrule{6-7}
		(source)& CoGenT-A set & Acc [\%]  & CoGenT-B set & Acc [\%]  & CoGenT set~ & Acc [\%] \\
		
\midrule				
& \multirow{5}{*}{Training Full?}   & \multirow{5}{*}{N/A}  & \multirow{2}{*}{--} & \multirow{2}{*}{--}  &   A Test?    &   96.6  \\
\cmidrule{6-7} 
PG+EE &   &    &   &    & B Test Full?    &   73.7  \\
\cmidrule{4-5}\cmidrule{6-7}
(\cite{johnson2017inferring}) &  &    & \multirow{2}{*}{B Train 30k?}  & \multirow{2}{*}{N/A}     & A Test?    &   76.1 \\
\cmidrule{6-7} 
&   &    &   &    & B Test Full?    &   92.7  \\
		
\midrule				
& \multirow{5}{*}{xxx}   & \multirow{5}{*}{N/A}  & \multirow{2}{*}{--} & \multirow{2}{*}{--}  &   AAA    &     \\
\cmidrule{6-7} 
MAC &   &    &   &    & BBB    &     \\
\cmidrule{4-5}\cmidrule{6-7}
(\cite{hudson2018compositional}) &  &    & \multirow{2}{*}{CoGenT-B}  & \multirow{2}{*}{N/A}     & CoGenT-A    &   \\
\cmidrule{6-7} 
&   &    &   &    & CoGenT-B    &    \\
		
		
\midrule				
& \multirow{5}{*}{A Train 670k}   & \multirow{5}{*}{xx}  & \multirow{2}{*}{--} & \multirow{2}{*}{--}  &   A Valid 30k    &     \\
\cmidrule{6-7} 
MAC &   &    &   &    & CoGenT-B    &     \\
\cmidrule{4-5}\cmidrule{6-7}
(ours) &  &    & \multirow{2}{*}{CoGenT-B 30k}  & \multirow{2}{*}{N/A}     & CoGenT-A    &   \\
\cmidrule{6-7} 
&   &    &   &    & CoGenT-B    &    \\
		
\midrule				
& \multirow{5}{*}{CoGenT-A}   & \multirow{5}{*}{N/A}  & \multirow{2}{*}{--} & \multirow{2}{*}{--}  &   CoGenT-A    &     \\
\cmidrule{6-7} 
S-MAC &   &    &   &    & CoGenT-B    &     \\
\cmidrule{4-5}\cmidrule{6-7}
(ours) &  &    & \multirow{2}{*}{CoGenT-B 30k}  & \multirow{2}{*}{N/A}     & CoGenT-A    &   \\
\cmidrule{6-7} 
&   &    &   &    & CoGenT-B    &    \\

		\bottomrule
	\end{tabular}
	\label{tab:generalization_comparison}
\end{table}


\subsection{The PG+EE model and training methodology}
The PG+EE (Program Generator and Execution Engine)~\cite{johnson2017inferring}  model is composed of two main modules:
a Program Generator constructing an explicit, graph-like representation of the reasoning process, and an Execution Engine executing that program and producing an answer. 
Both modules are implemented by neural networks, and were trained using a combination of backpropagation and REINFORCE~\cite{williams1992simple}.

The author inform that in the first step they trained their models on Condition A, and tested them on both Condition A and Condition B. 
Next, they fine-tuned these models on Condition B using 3K images and 30K questions, and again tested on both Conditions.
Sadly, it is not clear what sets exactly they used for fine-tuning (as Condition B contains only test set, thus lack training set) and testing on Condition A (as Condition A does not offer test set).
One possibility, as they are the authors of CLEVR and CoGenT datasets, is that they actually generated the missing sets, but didn't share them publicly.

\subsection{The FiLM model and training methodology}

\subsection{The TbD model and training methodology}

\subsection{Our MAC and S-MAC models and methodology}

Due to the fact that test set targets for both CLEVR and CoGenT aren't publicly available, the authors of~\cite{mascharka2018transparency} decided to split
the CoGenT validation sets and use part of then for training, and part for testing. 
We decided to follow that approach and ... \tk{Vincent, please proceed :)}  




\section{Illustration of failures of MAC on CLEVR}

\begin{itemize}
\item one with color failure -- that the attention is pointing at the right object, whereas the answer is improper because he never saw this combination
\item one with shape failure -- similarly	
\end{itemize}
Pictures with some description