\appendix

\section{Description of datasets}

\begin{enumerate}
\item Some description (with link to the paper) about the datasets. Maybe we can "import" material from main paper, making it lighter...
\item Table with attributes/feature combinations of CLEVR/CoGenT-A/CoGenT-B

\begin{table}
	\centering
	\begin{tabular}{ccccCcCc}
		\toprule
		Dataset        & Cubes              & Cylinders &  Spheres         \\
		\midrule
		CLEVR   &  any color &  any color        &    any color    \\
		%\midrule
		CLEVR CoGenT A & gray / blue / brown / yellow  & red / green / purple / cyan       &    any color  \\
		CLEVR CoGenT B  & red / green / purple / cyan &   gray / blue / brown / yellow       &      any color  \\
		\bottomrule
	\end{tabular}
	\caption{Comparison of the different colors/shapes combinations between CLEVR, CLEVR CoGenT-A and CLEVR CoGenT-B  }
	\label{tab:parameters}
\end{table}

\item Some exemplary pictures?
\end{enumerate}

\section{Full MAC and S-MAC comparison}

We think that in the main paper we should leave:
\begin{itemize}
\item first row - MAC - train on CLEVR, test on CLEVR 
\item all rows regaring S-MAC
\end{itemize}

When we will fill the whole table we will copy the results in here.

\section{Illustration of failures of MAC on CLEVR}

\begin{itemize}
\item one with color failure -- that the attention is pointing at the right object, whereas the answer is improper because he never saw this combination
\item one with shape failure -- similarly	
\end{itemize}
Pictures with some description