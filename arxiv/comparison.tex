\subsection{Comparison of generalization capabilities with other models}
\label{sec:comparison}

In this section, we present a comparison of our results on generalization capabilities with selected state-of-the-art models.
In particular, we focused on three papers reporting state-of-the-art accuracies on CoGenT. Those papers introduce the following models: PG+EE~\cite{johnson2017inferring}, FiLM~\cite{perez2017film} and TbD~\cite{mascharka2018transparency}.
Deeper analysis of those papers revealed that it is likely that different authors used different sets for reporting the scores, which questions the correctness of the comparison.
We find that the problem results from the fact that ground truth answers for the test sets are not provided along with these sets; thus subsets of the validation sets were sometimes used for testing. 
The results of our research are presented in \tableref{tab:generalization_comparison}, where we shortened the names of the datasets.
For instance, \textbf{A Test Full} means the use of the whole \textbf{CoGenT Condition A Test set}, whereas \textbf{B Valid 30k} indicates the use of 30.000 samples from \textbf{CoGenT Condition B Validation set}.
Question marks indicate that the paper does not provide enough information; thus the indicated sets are the ones we assumed were used.

\begin{table}[!h]
	\centering
	\begin{tabular}{cCcCcCc}
		\toprule
		Model & \multicolumn{2}{c}{Training} &    \multicolumn{2}{c}{Fine-tuning} &   \multicolumn{2}{c}{Test} \\		
		\cmidrule{2-3} \cmidrule{4-5}\cmidrule{6-7}
		(source)& CoGenT set & Acc [\%]  & CoGenT set & Acc [\%]  & CoGenT set~ & Acc [\%] \\
		
		\midrule				
		& \multirow{5}{*}{A Train Full?}   & \multirow{5}{*}{N/A}  & \multirow{2}{*}{--} & \multirow{2}{*}{--}  &   A Test Full    &   96.6  \\
		\cmidrule{6-7} 
		PG+EE &   &    &   &    & B Test Full?    &   73.7  \\
		\cmidrule{4-5}\cmidrule{6-7}
		(\cite{johnson2017inferring}) &  &    & \multirow{2}{*}{B Train 30k?}  & \multirow{2}{*}{N/A}     & A Test Full    &   76.1 \\
		\cmidrule{6-7} 
		&   &    &   &    & B Test Full    &   92.7  \\
		
		\midrule				
		CNN+GRU+FiLM & \multirow{5}{*}{A Train Full?}   & \multirow{5}{*}{N/A}  & \multirow{2}{*}{--} & \multirow{2}{*}{--}  &   A Valid Full?    &  98.3   \\
		\cmidrule{6-7} 
		0-Shot &   &    &   &    & B Valid 120k    &   78.8  \\
		\cmidrule{4-5}\cmidrule{6-7}
		(\cite{perez2017film}) &  &    & \multirow{2}{*}{B Valid 30k}  & \multirow{2}{*}{N/A}     & A Valid Full?    & 81.1  \\
		\cmidrule{6-7} 
		&   &    &   &    & B Valid 120k    &  96.9  \\
		
		
		\midrule				
		& \multirow{5}{*}{A Train Full}   & \multirow{5}{*}{N/A}  & \multirow{2}{*}{--} & \multirow{2}{*}{--}  &   A Test Full    &  98.8   \\
		\cmidrule{6-7} 
		TbD + reg &   &    &   &    & B Test Full    &  75.4   \\
		\cmidrule{4-5}\cmidrule{6-7}
		(\cite{mascharka2018transparency}) &  &    & \multirow{2}{*}{B Valid 30k}  & \multirow{2}{*}{N/A}     & A Test Full    &  96.9 \\
		\cmidrule{6-7} 
		&   &    &   &    & B Test Full   &  96.3  \\
		
		
		\midrule				
		& \multirow{5}{*}{A Train 630k}   & \multirow{5}{*}{97.02}  & \multirow{2}{*}{--} & \multirow{2}{*}{--}  &   A Valid Full    &     96.88 \\
		\cmidrule{6-7} 
		MAC &   &    &   &    & B Valid Full   &  79.54   \\
		\cmidrule{4-5}\cmidrule{6-7}
		(our results) &  &    & \multirow{2}{*}{B Valid 30k}  & \multirow{2}{*}{97.91}     & A Valid Full    &  92.06 \\
		\cmidrule{6-7} 
		&   &    &   &    & B Valid 120k    &   95.62 \\
		
		\midrule				
		& \multirow{5}{*}{A Train 630k}   & \multirow{5}{*}{96.09}  & \multirow{2}{*}{--} & \multirow{2}{*}{--}  &   A Valid Full    &     95.91 \\
		\cmidrule{6-7} 
		S-MAC &   &    &   &    & B Valid Full   &  78.71   \\
		\cmidrule{4-5}\cmidrule{6-7}
		(our results) &  &    & \multirow{2}{*}{B Valid 30k}  & \multirow{2}{*}{96.85}     & A Valid Full    &  91.24 \\
		\cmidrule{6-7} 
		&   &    &   &    & B Valid 120k    &   94.55 \\
		
		\bottomrule
	\end{tabular}
	\caption{Generalization capabilities of selected state-of-the-art models.}
	\label{tab:generalization_comparison}
\end{table}


\subsubsection{The PG+EE model and training methodology}
The PG+EE (Program Generator and Execution Engine)~\cite{johnson2017inferring} model is composed of two main modules:
a Program Generator constructing an explicit, graph-like representation of the reasoning process, and an Execution Engine executing that program and producing an answer. 
Both modules are implemented by neural networks, and were trained using a combination of backpropagation and REINFORCE~\cite{williams1992simple}.

The authors inform that in the first step they trained their models on Condition A, and tested them on both conditions. 
Next, they fine-tuned these models on Condition B using 3K images and 30K questions, and again tested on both conditions.
We could not ascertain which sets they used for fine-tuning (as Condition B lacks a training set).
Being the authors of the CLEVR and CoGenT datasets, it is possible that they generated specific sets. Additionally, as they own the ground truth answers for the test sets, we take as granted that they reported the accuracies on both Condition A and Condition B test sets, despite they explicitly didn't say that in the paper.

\subsubsection{The FiLM model and training methodology}

Feature-wise Linear Modulation (in short, FiLM)~\cite{perez2017film} is an optional enhancement of a neural network model.
The idea is to influence the behavior of existing layer(s) by introducing feature-wise affine transformations which are conditioned on the input.
A model composed of CNNs and a GRU with FiLM-enhanced layers achieved state-of-the-art results on both CLEVR and CoGenT, showing improvement over the PG+EE model.

Nonetheless, the authors indicate in the paper that the accuracy reported for Condition B after fine-tuning was obtained on the CoGenT Condition B Validation set, excluding the 30k samples which were used for fine-tuning.
This suggests that they probably reported scores for Condition A on the validation, not test set as well.


\subsubsection{The TbD model and training methodology}

The TbD (Transparency by Design) network was introduced in~\cite{mascharka2018transparency}.
TbD is composed of a set of visual reasoning primitives relying on attention transformations, allowing the model to perform reasoning by composing attention masks.
The authors compare the accuracy of their model tested on CLEVR, alongside with several existing models. In particular, they show improvement over the previously mentioned PG + EE, CNN + GRU + FiLM and MAC models.

They also compare their performance on CoGenT against PG+EE.
The authors (in a separate conversation) explained that they trained their models on Condition A training set and tested on Condition A and Condition B test sets respectively.
For fine-tuning on CoGenT-B the authors used the first 3k images (Val-B indices 0-2999) in the CoGenT-B validation dataset and all of their corresponding questions.



They indicate that they used 3k images and 30k questions from the CoGenT Condition B for fine-tuning of their model, but do not explicitly point the set they used the samples from.


\subsubsection{Our MAC and S-MAC models and methodology}

Due to the fact that the test sets ground truth labels for both CLEVR and CoGenT aren't publicly available, we decided to follow the approach proposed in~\cite{perez2017film} and split the CoGenT B Validation set.
As a result we used 30k samples for fine-tuning and the remaining ones for testing.
When testing the model on CoGenT A, we used the whole validation set.

Moreover, as we were using the validation sets for testing, we also splitted the training set and used 90\% for training and the remaining 10\% for validation during training.
We used the same methodology for all the results presented in this paper, i.e. \tableref{tab:results}, \tableref{tab:generalization_comparison} and \tableref{tab:results_full}.
