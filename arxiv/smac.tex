\subsection{Simplified MAC network}
Our proposed modification to the MAC network is based on two heuristic
simplifications of the MAC cell. 
First, we observe that, taking the MAC cell equations as a whole, consecutive linear layers (with no activation in-between) can be combined as one linear layer.
Next, we assume that dimension-preserving linear layers are invertible
so as to avoid information loss. 
Applying this principle to the equations, with a careful reorganization,
we can apply a single linear layer to the knowledge base \emph{prior}
to all the reasoning steps and work with this \emph{projected} knowledge base
as-is throughout the reasoning steps.


%Applying this principle to 
%equations in the MAC cell results in eliminating one of the linear layers applied to
%the knowledge base. Having reduced the number of linear projections on the knowledge base to one, we can, with a careful reorganization, compute it prior to the reasoning steps. Thus, we can work purely
%with this \emph{projected} knowledge base. 

\noindent\textit{Notation.}
%Let $q$ denote the global context of the question
%and let $\cw_s$ denote the contextual embedding of a word $s$ in the question.
For each knowledge base $\kb$ of dimension $H \times W \times d$, let $\kb_{hw}$ be the $d$-dimensional vector indexed by 
$h \in \{1,2,\dots,H\}$ and $w \in \{1,2,\dots, W\}$.
Let $i$ denote the index of the reasoning steps. 

In the description below, the original MAC cell equations are shown on the \emph{left}
while our simplified equations are shown (in color) on the {\color{Plum} \emph{right}}.
The equation numbering is the same as in~\cite{hudson2018compositional}.

\noindent\textbf{Control unit:} 
For both models, in the control unit, the question $q$ is first transformed in each step of 
the reasoning using a \emph{position-aware}
linear layer depending on $i$: $q_i = U_i^{[d \times 2d]} q + b_i^{[d]}$.

\begin{multicols}{2}
	\noindent
	\begin{align*}
	&cq_i = W_{cq}^{[d \times 2d]} [c_{i-1}, q_i] + b_{cq}^{[d]}  \tag{c1} \\
	&ca_{is} = W_{ca}^{[1 \times d]} (cq_i \odot \cw_s) + b_{ca}^{[1]}
	\tag{c2.1}\\
	&cv_{is} = \textrm{softmax}(ca_{is}) \tag{c2.2}\\
	&\cc_i = \sum_s cv_{is} \, \cw_s  \tag{c2.3}
	\end{align*}
	\columnbreak
	{\color{Plum}
	\begin{align*}
	&cq_i = W_{cq}^{[d \times d]} c_{i-1} + q_i  \tag{c1} \\
	&ca_{is} = W_{ca}^{[1 \times d]} (cq_i \odot \cw_s)  \tag{c2.1}\\
	&cv_{is} = \textrm{softmax}(ca_{is}) \tag{c2.2}\\
	&\cc_i = \sum_s cv_{is} \, \cw_s  \tag{c2.3}
    \end{align*}}
\end{multicols}

\vskip -0.6cm
\noindent\textbf{Read and write units:}
\begin{multicols}{2}
	\noindent
	\begin{align*}
	&I_{ihw} = (W_{m}^{[d \times d]} \mem_{i-1} + b_{m}^{[d]}) \\
	           & \qquad \quad \odot (W_{k}^{[d \times d]} \kb_{hw} + b_{k}^{[d]}) \tag{r1} \\
	&I'_{ihw} =  W_{I'}^{[d \times 2d]} [I_{ihw},\kb_{hw}]  + b_{I'}^{[d]}  \tag{r2} \\
	&ra_{ihw} = W_{ra}^{[1 \times d]} (\cc_i \odot I'_{ihw}) + b_{ra}^{[1]} \tag{r3.1}\\
	&rv_{ihw} = \textrm{softmax}(ra_{ihw}) \tag{r3.2}\\
	&\rr_i = \sum_s rv_{ihw} \, \kb_{hw}  \tag{r3.3}\\
	&\mem_i = W_{rm}^{[d \times 2d]} [\rr_i, \mem_{i-1}]  + b_{rm}^{[d]} \tag{w1}	
	\end{align*}
	\columnbreak
	{\color{Plum}
	\begin{align*}
	&I_{ihw} = \mem_{i-1} \odot \kb_{hw} \tag{r1} \\ \\
	&I'_{ihw} = W_{I'}^{[d \times d]} I_{ihw} + b_{I'}^{[d]} + \kb_{hw} \tag{r2} \\
	&ra_{ihw} = W_{ra}^{[1 \times d]} (\cc_i \odot I'_{ihw})  \tag{r3.1}\\
	&rv_{ihw} = \textrm{softmax}(ra_{ihw}) \tag{r3.2}\\
	&\rr_i = \sum_s rv_{ihw} \, \kb_{hw}  \tag{r3.3}\\
	&\mem_i = W_{rm}^{[d \times d]} \rr_i + b_{rm}^{[d]} \tag{w1}
	\end{align*}}
\end{multicols}

As seen above, being noticeably simpler, the S-MAC obtains significant reduction in the number of position-independent parameters across all units (see~\tableref{tab:parameters}).
Our experiments demonstrate that this gives us noticeable savings in the training time. However, since the computation time is also dominated by the position-aware layers in the control unit, as well as the input unit, the speedup is not as large as we desire.


\begin{table}[!t]
	\centering
	\begin{tabular}{ccccCcCc}
		\toprule
		Model        & Read Unit               & Write Unit &  Control Unit         \\
		\midrule
		MAC   &  787,969 &  524,800        &    525,313    \\
		%\midrule
		simplified MAC & 263,168  & 262,656       &    263,168 \\
		\midrule
		Reduction by [\%]  & 67\%  &   50\%       &      50\%  \\
		\bottomrule
	\end{tabular}
	\caption{Comparison of the number of position-independent parameters in MAC \& S-MAC cells.}
	\label{tab:parameters}
\end{table}

